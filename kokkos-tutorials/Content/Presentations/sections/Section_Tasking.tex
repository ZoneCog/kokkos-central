
%==========================================================================

\begin{frame}[fragile]

  {\Huge Task parallelism}

  \vspace{10pt}

  {\large Fine-grained dependent execution.}

  \vspace{20pt}

  \textbf{Learning objectives:}
  \begin{itemize}
    \item {Basic interface for fine-grained tasking in Kokkos}
    \item {How to express dynamic dependency structures in Kokkos tasking}
    \item {When to use Kokkos tasking}
  \end{itemize}

  \vspace{-20pt}

\end{frame}

%==========================================================================

\begin{frame}[fragile]{Task Parallelism Looks Like Data Parallelism}

    Recall that {\textbf{data parallel}} code is composed of a {\color{patternColor!80!black} pattern}, a {\color{policyColor!80!black} policy}, and a {\color{bodyColor!80!black} functor}

    \vspace{3pt}

    \begin{code}[linebackgroundcolor={},keywords={}]
@patternKokkos::parallel_for@pattern(
  @policyKokkos::RangePolicy<>(exec_space, 0, N)@policy,
  @bodySomeFunctor()@body
);
  \end{code}
    \vspace{8pt}

    \textbf{Task parallel} code similarly has a {\color{patternColor!80!black} pattern}, a {\color{policyColor!80!black} policy}, and a {\color{bodyColor!80!black} functor}

    \vspace{3pt}

    \begin{code}[linebackgroundcolor={},keywords={}]
@patternKokkos::task_spawn@pattern(
  @policyKokkos::TaskSingle(scheduler, TaskPriority::High)@policy,
  @bodySomeFunctor()@body
);
    \end{code}
\end{frame}

%==========================================================================

\begin{frame}[fragile]{What does a task functor look like?}

  \begin{code}[linebackgroundcolor={},keywords={}]
struct MyTask {
  using @bluevalue_type@blue = @reddouble@red;
  template <class @darkgreenTeamMember@darkgreen>
  KOKKOS_INLINE_FUNCTION
  void operator()(@darkgreenTeamMember@darkgreen& member, @reddouble@red& result);
};
  \end{code}
  
  \begin{itemize}
    \item Tell Kokkos what the {\color{blue}value type} of your task's output is.
    \item Take a {\color{darkgreen}team member} argument, analogous to the team member passed in by \texttt{Kokkos::TeamPolicy} in hierarchical parallelism
    \item The {\color{red} output} is expressed by assigning to a parameter, similar to with \texttt{Kokkos::parallel\_reduce}
  \end{itemize}

\end{frame}

%==========================================================================

\begin{frame}[fragile]{What policies does Kokkos tasking provide?}

  \begin{itemize}
    \item \texttt{Kokkos::TaskSingle()}
      \begin{itemize}
        \item Run the task with a single worker thread
      \end{itemize}
    \item \texttt{Kokkos::TaskTeam()}
      \begin{itemize}
        \item Run the task with all of the threads in a team
        \item Think of it like being inside of a \texttt{parallel\_for} with a \texttt{TeamPolicy}
      \end{itemize}
    \item Both policies take a scheduler, an optional predecessor, and an optional priority (more on schedulers and predecessors later)
  \end{itemize}

\end{frame}

%==========================================================================

\begin{frame}[fragile]{What patterns does Kokkos tasking provide?}

  \begin{itemize}
    \item \texttt{Kokkos::task\_spawn()}
    \begin{itemize}
      \item \texttt{Kokkos::host\_spawn()} (same thing, but from host code)
    \end{itemize}
  \item \texttt{Kokkos::respawn()}
    \begin{itemize}
      \item {\color{red}Argument order is backwards; policy comes second!}
      \item {\color{red}First argument is `this` always (not `*this`)}
    \end{itemize}
  \item \texttt{task\_spawn()} and \texttt{host\_spawn()} return a \texttt{Kokkos::Future} representing the completion of the task (see next slide), which can be used as a predecessor to another operation.
  \end{itemize}

\end{frame}

%==========================================================================

\begin{frame}[fragile]{How do futures and dependencies work?}

  \begin{code}[linebackgroundcolor={},keywords={}]
@graystruct MyTask {
  using value_type = double;@gray
  Kokkos::Future<double, Kokkos::DefaultExecutionSpace> @bluedep@blue;
  int @darkgreendepth@darkgreen;
  @grayKOKKOS_INLINE_FUNCTION MyTask(int d) : depth(d) { } 
  template <class TeamMember>
  KOKKOS_INLINE_FUNCTION
  void operator()(TeamMember& member, double& result)@gray {
    if(@darkgreendepth@darkgreen == 1) result = 3.14;
    else if(@bluedep@blue.is_null()) {
      @bluedep@blue =
        @darkredKokkos::task_spawn(
          Kokkos::TaskSingle(member.scheduler()),
          MyTask(@darkred@darkgreendepth@darkgreen@darkred-1)
        );@darkred
      Kokkos::respawn(this, @bluedep@blue);
    }
    else {
      result = @darkgreendepth@darkgreen * @bluedep@blue.get();
    }
  }
};
  \end{code}
\end{frame}

%==========================================================================

\begin{frame}[fragile]{The Scheduler Abstraction}
  \begin{code}[keywords={}]
template <class @blueScheduler@blue>
@graystruct MyTask {
  using value_type = double;@gray
  Kokkos::BasicFuture<double, @blueScheduler@blue> @graydep;
  int depth;
  KOKKOS_INLINE_FUNCTION MyTask(int d) : depth(d) { } 
  template <class TeamMember>
  KOKKOS_INLINE_FUNCTION
  void operator()(TeamMember& member, double& result);
};@gray
  \end{code}

    \vspace{1em}

  {\em{Available Schedulers:}}
  \begin{itemize}
    \item \texttt{TaskScheduler<ExecSpace>}
    \item \texttt{TaskSchedulerMultiple<ExecSpace>}
    \item \texttt{ChaseLevTaskScheduler<ExecSpace>}
  \end{itemize}
\end{frame}

%==========================================================================

\begin{frame}[fragile]{Spawning from the host}
  \begin{code}[keywords={}]
using execution_space = Kokkos::DefaultExecutionSpace;
using scheduler_type = Kokkos::TaskScheduler<execution_space>;
using memory_space = scheduler_type::memory_space;
using memory_pool_type = scheduler_type::memory_pool;
size_t memory_pool_size = 1 << 22;

auto @bluescheduler@blue = 
  scheduler_type(memory_pool_type(memory_pool_size));

Kokkos::BasicFuture<double, scheduler_type> @darkgreenresult@darkgreen =
  @darkredKokkos::host_spawn(
    Kokkos::TaskSingle(@bluescheduler@blue),
    MyTask<scheduler_type>(10)
  );@darkred
Kokkos::wait(@bluescheduler@blue);
printf("Result is %f", @darkgreenresult@darkgreen.get());
  \end{code}
\end{frame}

%==========================================================================

\begin{frame}[fragile]{Things to Keep in Mind}
  \begin{itemize}
    \item Tasks always run to completion
    \item There is no way to wait or block inside of a task
      \begin{itemize}
        \item {\color{red} \texttt{future.get()} does not block!}
      \end{itemize}
    \item Tasks that do not \texttt{respawn} themselves are complete
      \begin{itemize}
        \item The value in the \texttt{result} parameter is made available through \texttt{future.get()} to any dependent tasks.
      \end{itemize}
    \item The second argument to \texttt{respawn} can only be either a predecessor (future) or a scheduler, not a proper execution policy
      \begin{itemize}
        \item We are fixing this to provide a more consistent overload in the next release.
      \end{itemize}
    \item Tasks can only have one predecessor (at a time)
      \begin{itemize}
        \item Use \texttt{scheduler.when\_all()} to aggregate predecessors (see next slide)
      \end{itemize}
  \end{itemize}
\end{frame}

%==========================================================================

\begin{frame}[fragile]{Aggregate Predecessors}
  \begin{code}[keywords={}]
    using @purplevoid_future@purple =
      Kokkos::BasicFuture<void, scheduler_type>;
    auto @darkredf1@darkred =
      Kokkos::task_spawn(Kokkos::TaskSingle(scheduler), X{});
    auto @darkgreenf2@darkgreen =
      Kokkos::task_spawn(Kokkos::TaskSingle(scheduler), Y{});
    @purplevoid_future@purple @orangef_array@orange[] = { @darkredf1@darkred, @darkgreenf2@darkgreen };
    @purplevoid_future@purple @redf_12@red = scheduler.@bluewhen_all@blue(@orangef_array@orange, 2);
    auto f3 =
      Kokkos::task_spawn(
        Kokkos::TaskSingle(scheduler, @redf_12@red), FuncXY{}
      );
  \end{code}
  \begin{itemize}
    \item To create an aggregate \texttt{Future}, use \texttt{scheduler.when\_all()}
    \item \texttt{scheduler.when\_all()} always returns a \texttt{void} future.
    \item (Also, any future is implicitly convertible to a \texttt{void} future of the same \texttt{Scheduler} type)
  \end{itemize}
\end{frame}


%==========================================================================

\begin{frame}[fragile]{Exercise: Fibonacci}
    {\ul{\textbf{Naïve Recursive Fibonacci}}}

  \begin{columns}[t,onlytextwidth]
    \column{.45\textwidth}
      \begin{center}
          \vspace{-2em}
          {\ul{\textit{Formula}}}
\begin{align*}
    F_N &= F_{N-1} + F_{N-2} \\
    F_0 &= 0 \\
    F_1 &= 1
\end{align*}
      \end{center}
    \column{.55\textwidth}
      {\ul{\textit{Serial algorithm}}}
\begin{code}[keywords={}]
int @darkgreenfib@darkgreen(int @darkredn@darkred) {
  if(@darkredn@darkred < 2) return @darkredn@darkred;
  else {
    return @darkgreenfib@darkgreen(@darkredn@darkred-1) + @darkgreenfib@darkgreen(@darkredn@darkred-2);
  }
}
\end{code}
  \end{columns}

    {\textbf{Details:}}
    \begin{itemize}
        \item Location: \texttt{Exercises/tasking}
        \item Implement the \texttt{FibonacciTask} task functor recursively
        \item Spawn the root task from the host and wait for the scheduler to make it ready
    \end{itemize}
    {\textbf{Hints:}}
    \begin{itemize}
        \item Do the $F_{N-1}$ and $F_{N-2}$ subproblems in separate tasks
        \item Use a \texttt{scheduler.when\_all()} to wait on the subproblems
    \end{itemize}
\end{frame}

%==========================================================================
