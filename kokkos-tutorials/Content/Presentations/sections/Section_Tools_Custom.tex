\begin{frame}[fragile]

  {\Huge Custom Tools}

  \vspace{10pt}

  {\large How to write your own tools for the KokkosP interface.}

  \vspace{20pt}

  \textbf{Learning objectives:}
  \begin{itemize}
    \item {The KokkosP hooks}
    \item {Callback registration inside the application}
    \item {Throwaway debugging tools}
  \end{itemize}

  \vspace{-20pt}

\end{frame}

%==========================================================================

\begin{frame}[fragile]{Motivation}

KokkosTools also allow you to write your own tools!

\begin{itemize}
  \item Implement a simple C interface.
  \item Only implement what you want to use!
  \item Full access to the entire instrumentation.
\end{itemize}

\vspace{10pt}
But why would I want to do that?

\begin{itemize}
  \item Profiling tools which know about your code structure and properly categorize information.
  \item Add in situ analysis hooked into your CI system.
  \item Write debugging tools specific for your framework.
  \item Write throwaway debugging tools for larger apps, instead of recompiling.
\end{itemize}

\pause
\textit{We will first walk through the hooks and then illustate with an example.}
\end{frame}

%==========================================================================

\begin{frame}[fragile]{Infrastructure and Initialization}
\textbf{Some Helper Classes}
\begin{code}
// Contains a unique device identifier.
struct KokkosPDeviceInfo { uint32_t deviceID; };

// Unique name of execution and memory spaces.
struct SpaceHandle { char name[64]; };
\end{code}

\pause
\textbf{Initialization and Finalization hooks}
\begin{code}[keywords={extern,void,int,uint64_t,uint32_t}]
extern "C" void kokkosp_init_library(
  int loadseq, uint64_t version, uint32_t num_devinfos,
  KokkosPDeviceInfo* devinfos);
\end{code}
\vspace{-15pt}
  \begin{itemize}
    \item Called during \texttt{Kokkos::initialize}
    \item Provides device ids used subsequently.
    \item Use this call to setup tool infrastructure.
  \end{itemize}
\begin{code}[keywords={extern,void,int,uint64_t,uint32_t}]
extern "C" void kokkosp_finalize_library();
\end{code}
\vspace{-15pt}
  \begin{itemize}
    \item Called during \texttt{Kokkos::finalize}
    \item Usually used to output results.
  \end{itemize}
\end{frame}

%==========================================================================

\begin{frame}[fragile]{Parallelism Hooks}
\begin{code}[keywords={extern,void,int,uint64_t,uint32_t,const,char}]
extern "C" {
  void kokkosp_begin_parallel_for(const char* name,
                                  uint32_t devid,
                                  uint64_t* kernid);
  void kokkosp_begin_parallel_reduce(const char* name,
                                     uint32_t devid,
                                     uint64_t* kernid);
  void kokkosp_begin_parallel_scan(const char* name,
                                   uint32_t devid,
                                   uint64_t* kernid);
};
\end{code}
\vspace{-15pt}
\begin{itemize}
  \item Called when a parallel dispatch is initiated.
  \item \texttt{name} is the user provided string or a typeid.
  \item \texttt{kernid} is set by the tool to match up with the \texttt{end} call.
\end{itemize}

\begin{code}[keywords={extern,void,int,uint64_t,uint32_t,const,char}]
extern "C" void kokkosp_end_parallel_for(uint64_t kernid);
extern "C" void kokkosp_end_parallel_reduce(uint64_t kernid);
extern "C" void kokkosp_end_parallel_scan(uint64_t kernid);
\end{code}
\vspace{-15pt}
\begin{itemize}
  \item Called when a parallel dispatch is done.
  \item \texttt{kernid} is the value the begin call set.
\end{itemize}
\end{frame}

%==========================================================================

\begin{frame}[fragile]{Memory Hooks}
\begin{code}[keywords={extern,void,int,uint64_t,uint32_t,const,char}]
extern "C" void kokkosp_begin_deep_copy(
  SpaceHandle dst_hndl, const char* dst_name, const void* dst_ptr,
  SpaceHandle src_hndl, const char* src_name, const void* src_ptr,
  uint64_t size);
\end{code}
\begin{itemize}
  \item Called when a \texttt{deep\_copy} is started.
  \item Provides space handles, names, ptrs and size of allocations.
\end{itemize}

\begin{code}[keywords={extern,void,int,uint64_t,uint32_t,const,char}]
extern "C" void kokkosp_end_deep_copy();
\end{code}
\begin{itemize}
  \item Called when a \texttt{deep\_copy} is done.
\end{itemize}

\begin{code}[keywords={extern,void,int,uint64_t,uint32_t,const,char}]
extern "C" void kokkosp_allocate_data(SpaceHandle hndl,
  const char* name, void* ptr, uint64_t size);
extern "C" void kokkosp_deallocate_data(SpaceHandle hndl,
  const char* name, void* ptr, uint64_t size);
\end{code}
\begin{itemize}
  \item Called when allocating or deallocating data.
\end{itemize}
\end{frame}

%==========================================================================

\begin{frame}[fragile]{Callback Registration}
Sometimes it is useful to build a tool into an executable.

\vspace{5pt}
\begin{block}{Callback Registration}
Kokkos Tools provide a callback setting system to set tool callbacks from within the application.
\end{block}

Takes the form of:
\begin{code}
void set_HOOK_callback(HOOK_FUNCTION_PTR callback);
\end{code}

Where \texttt{HOOK} is one of 
\begin{code}
init finalize push_region pop_region begin_parallel_for 
end_parallel_for begin_parallel_reduce end_parallel_reduce
begin_parallel_scan end_parallel_scan begin_fence end_fence
allocate_data deallocate_data begin_deep_copy end_deep_copy
\end{code}

One can also store a callback set, reload it and pause tool calls
\begin{code}
EventSet get_callbacks();  void set_callbacks(EventSet);
void pause_tools();        void resume_tools();
\end{code}
\end{frame}

\begin{frame}[fragile]{Callback Registration}
Example:
\begin{lstlisting}[language=C++]
#include <Kokkos_Core.hpp>
using Kokkos::Profiling;
using Kokkos::Tools::Experimental;
using Kokkos;

void kokkosp_allocate_data(SpaceHandle space,
  const char* label, const void* const ptr, uint64_t size) {
  printf("Allocate: [%s] %lu\n",label,size);
}
void kokkosp_deallocate_data(SpaceHandle space,
  const char* label, const void* const ptr, uint64_t size) {
  printf("Deallocate: [%s] %lu\n",label,size);
}

int main(int argc, char* argv[]) {
  initialize(argc, argv);
  set_allocate_data_callback(kokkosp_allocate_data);
  set_deallocate_data_callback(kokkosp_deallocate_data);
  ...
  finalize();
}
\end{lstlisting}

\end{frame}

%==========================================================================

\begin{frame}[fragile]{Example: Throwaway Debugging Tool}
Sometimes you just need to know what is in a View before and after entering a kernel for the 5th time:
\begin{itemize}
  \item The view is on the GPU and its on some rank of a large run.
  \item Recompiling the app takes hours.
\end{itemize}

\pause
\vspace{10pt}
\textbf{Simple Kokkos tool could do it!}

What we need:
\begin{itemize}
  \item Store the pointer and size of the view with a specific label when it gets allocated.
  \item Print the View when entering a kernel and before exiting it. 
  \item Make sure the view didn't get deallocated in the mean time. 
\end{itemize}
\end{frame}

\begin{frame}[fragile]{Example: Throwaway Debugging Tool}
Store the pointer:
\vspace{-2pt}
\begin{code}[keywords={int,uint64_t,extern,void,const,char,void,if}]
int* data; uint64_t N; int count;
extern "C" void kokkosp_allocate_data(SpaceHandle handle,
  const char* name, void* ptr, uint64_t size) {
  if(strcmp(name,"PuppyWeights")==0) {
    data = (int*)ptr+32; N = size; count = 0;
}}
\end{code}

Print the View:
\vspace{-2pt}
\begin{code}[keywords={int,uint64_t,extern,void,const,char,void,if,else}]
void print_data() { 
  std::vector<int> hcpy(N);
  cudaMemcpy(hcpy.data(),data,N*sizeof(int));
  for(int i=0;i<N;++i) printf("(%d %d)",i,hcpy[i]); printf("\n");
}
extern "C" void kokkosp_begin_parallel_for(const char* name, 
  uint32_t, uint64_t* kernid) {
  if(strcmp(name,"PuppyOnCouch")==0) { 
    count++; if(count==5) print_data(); *kernid=1;
  } else { *kernid = 0; } 
}
extern "C" void kokkosp_end_parallel_for(uint64_t kernid) {
  if(kernid == 1 && count==5) print_data();
}
\end{code}

\end{frame}

\begin{frame}[fragile]{TestCode}
\begin{code}[keywords={int,uint64_t,extern,void,const,char,void,parallel_for,for,initialize,finalize}]
#include <Kokkos_Core.hpp>
#include <cmath>

int main(int argc, char* argv[]) {
  Kokkos::initialize(argc, argv);
  {
    int N = argc > 1 ? atoi(argv[1]) : 12;
    int R = argc > 2 ? atoi(argv[2]) : 10;
    Kokkos::View<double*> a("PuppyWeights",N);

    for(int r=0; r<R; r++) {
      Kokkos::parallel_for("PuppyOnCouch",N,KOKKOS_LAMBDA(int i)
                           { a(i) = i*r; });
    }
  }
  Kokkos::finalize();
}
\end{code}

Output:
\begin{code}
(0 0) (1 4) (2 8) (3 12)
(0 0) (1 5) (2 10) (3 15)
\end{code}
\end{frame}

\begin{frame}[fragile]{Hooks Summary}
\textbf{Implementing your own tools is easy!}
\begin{itemize}
  \item Simply implement the needed C callback functions.
  \item Only implement what you need.
  \item Goal is to make it simple enough so that one-off tools are a viable debugging help.
\end{itemize}

\vspace{10pt}
\textbf{Callback registration for applications}
\begin{itemize}
  \item The callback registration system allows to embed tools in applications.
  \item Store callback sets and restore them.
\end{itemize}

\end{frame}

